\documentclass[10pt,t,usenames,dvipsnames]{beamer}
\usetheme{metropolis}           % Use metropolis theme

\ifnotes
  \hypersetup{final}
  \usepackage{pgfpages}
  \setbeamertemplate{note page}[plain]
  \setbeameroption{show notes on second screen=right}
  % the following fixes bug which causes normal text to be white instead of
  % template default when notes are enabled
  % (see https://tex.stackexchange.com/questions/232168/normal-text-is-invisible-when-using-beamer-with-notes-and-xelatex)
  \makeatletter 
  \def\beamer@framenotesbegin{% at beginning of slide
    \usebeamercolor[fg]{normal text}
    \gdef\beamer@noteitems{}% 
    \gdef\beamer@notes{}% 
  }
  \makeatother
\fi

\usepackage{appendixnumberbeamer}
\usepackage[scale=3]{ccicons}   % creative commons icons
\usepackage{verbatim}

\setbeamertemplate{itemize item}{}

\title{CSCI 338 --- Algorithm Design and Analysis}
\date{}
\author{Jeremy Iverson}
\institute{College of Saint Benedict \& Saint John's University}
\begin{document}
  \maketitle

  \begin{frame}{logistics}
    \begin{itemize}
      \setlength\itemsep{10pt}
      \item instructor
        \begin{itemize}
          \item Jeremy Iverson (\href{mailto:jiverson002@csbsju.edu}{\nolinkurl{jiverson002@csbsju.edu}})
          \item PENGL 213, (320) 363-3083
          \item office hours: TBD
        \end{itemize}
      \item textbook
        \begin{itemize}
          \item \emph{Introduction to the Design and Analysis of Algorithms}, 3rd Edition, Levitin
        \end{itemize}
      \item website
        \begin{itemize}
          \item \url{https://csbsju.instructure.com/courses/11420}
        \end{itemize}
    \end{itemize}

    \note{
      \begin{itemize}
        \item I prefer to be called Jeremy
        \item Encourage questions right away
        \item Emphasize the importance of the Canvas site for finding
          information about the class
        \item office hours
          \begin{itemize}
            \item Mention outlook calendar \& my home page
              \begin{itemize}
                \item For those unfamiliar with Outlook meetings, then they
                  should schedule another way and we will go over this in
                  meeting
              \end{itemize}
          \end{itemize}
        \item go through Canvas page organization quickly
      \end{itemize}
    }
  \end{frame}

  \begin{frame}{objectives}
    \begin{itemize}
      \setlength\itemsep{10pt}
      \item \textbf<2>{to use different abstract methodologies to construct
        algorithms that solve given problems}
      \item \textbf<3>{to analyze the time and space complexity of algorithms
        and compare the complexity of different algorithms}
      \item \textbf<4>{to describe the complexity classes 𝒫 and 𝒩𝒫 and explain
        why they are important in understanding computational tractability}
      \item \textbf<5>{to write a good description and analysis of an algorithm}
    \end{itemize}
  \end{frame}

  \begin{frame}{evaluation}
    \begin{itemize}
      \setlength\itemsep{10pt}
      \item eight (8) written assignments\\three (3) programming assignments
        \begin{itemize}%[label=\cdot]
          \item do your own work
          \item use each other as resources only
          \item due dates are strict (no partial credit for late assignments)
        \end{itemize}
      \item three (3) topical exams and one (1) cumulative
        \begin{itemize}%[label=\cdot]
          \item closed book / open note
        \end{itemize}
      \item point distribution
        \begin{itemize}
          \item written assignments: 3\% each
          \item programming assignments: 4\% each
          \item exams: 13\% each
          \item final: 25\%
        \end{itemize}
    \end{itemize}

    \note{Should be student's own work, may work together but must individual
      solutions\\}
    \note{Want to encourage using each other as resources, but do not want some
      students to rely on other students for answers}
  \end{frame}

  \begin{frame}[standout]{\mbox{need more info\dots}}
    \ifnotes
      \usebeamercolor[white]{normal text} % override bug fix in preamble
    \fi
    see~\href{https://csbsju.instructure.com/courses/11790/assignments/syllabus}{\nolinkurl{course} \nolinkurl{syllabus}}!
  \end{frame}

  \begin{frame}{a good question}
    \begin{block}{\color{OliveGreen}why bother?}
      \begin{itemize}
        \item better debugger
        \item better programmer
        \item better problem-solver
      \end{itemize}
    \end{block}

    \note{
      most of you will neither write nor maintain compilers nor will you program
      in assembly on a regular basis, so then why bother\dots

      \begin{itemize}
        \setlength\itemsep{10pt}
        \item better debugger --- improve tracing skill and have a better idea
          of how things you write are understood by the machine
        \item better programmer --- debugging is a skill that many
          ``programmers'' lack, but you will have improved this skill therefore,
          you will be a better programmer
        \item better problem-solver --- because of your new understanding of the
          orgranization of a computer system you will better understand its
          capabilities and limitations --- you will also gain experience moving
          between levels of abstraction, thus improving your abstract thinking
      \end{itemize}
    }
  \end{frame}

  \begin{frame}{a look ahead}
    \begin{itemize}
      \setlength\itemsep{10pt}
      \item \textbf<2>{develop a framework for analyzing algorithms}
      \item \textbf<3>{study some of the fundamental algorithm design
        techniques}
      \item \textbf<3>{look at a sample of the important algorithms in CS}
    \end{itemize}

    \only<2>{\note{we need to be able to reason about the algorithms that we
      design. we need some way to compare different algorithms that solve the
      same problem to understand which is better and when\dots bubble sort is
      easy to understand and implement, however, it is only the right best
      sorting algorithm under very specific circumstances.}}
    \only<3>{\note{algorithm design is not a process that starts from a blank
      slate every time --- complex problem-solving often relies on building
      blocks or reformulating one problem as another --- we will be learning
      some of these building blocks}}
  \end{frame}

  \begin{frame}{activity}
    \begin{enumerate}
      \setlength\itemsep{10pt}
      \item which is better and why, bubble sort or merge sort?
      \pause
      \item is there any circumstance when bubble sort is better than merge
        sort?
    \end{enumerate}
  \end{frame}

  \begin{frame}{a bit of parting advice}
    \begin{itemize}
      \item remember that this is a 300-level course
      \pause
      \item if something is confusing, tell me
    \end{itemize}

    \only<1->{\note{I expect you to do some learning on your own and come to
      class prepared to enhance / clarify / correct the understanding the you
      created. I want to help you explore your understanidng of the material,
      not necessarily help you gain a basic understanding of the material\\}}
    \only<2->{\note{otherwise, I am relying on my own experience to decide which
      material to emphasize and/or clarify\\[1\baselineskip]}}
  \end{frame}

  \begin{frame}{teaching philosophy}
    Your job is to empower those you teach; when you do for them what they
    should be doing for themselves, you create dependency rather than
    empowerment.

    It is easy to give in to the frustration that results from seeing amazing
    possibilities for the people you are teaching, and you want it more for them
    than they want it for themselves. 

    Don’t give in to that frustration! \\
    \hfill \footnotesize{--- Based on passage from ``Resisting Happiness'' by
      Matthew Kelly}

    \note{I sincerely believe that all of you can master this material and I
      want more than anything else this semester, to help you do that!}
    \note{However, don't expect me to give you answers. And do get frusterated
      with me when I challenge you to discover the answers for yourself.}
  \end{frame}

  \begin{frame}[standout]
    \ifnotes
      \usebeamercolor[white]{normal text} % override bug fix in preamble
    \fi
    questions?
  \end{frame}

  \appendix

  \begin{frame}{activity}
    \begin{itemize}
      \item download this slide deck and follow the instructions on the next
        slide
    \end{itemize}

    \note{if there is still time, talk about all of the material for this course
      being hosted on GitHub, and how they can access it}
  \end{frame}

  \ifnotes\else
  \begin{frame}{next slide}
    \begin{itemize}
      \item find out what the reading is for Wednesday
    \end{itemize}
  \end{frame}
  \fi

  \begin{frame}[c]
    \begin{center}\ccbysa\end{center}

    except where otherwise noted, this worked is licensed under
    \href{http://creativecommons.org/licenses/by-sa/4.0/}{creative commons
    attribution-sharealike 4.0 international license}
  \end{frame}
\end{document}
